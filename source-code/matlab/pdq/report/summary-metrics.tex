% Data generated by pdq_generate_reports:generate_summary_metrics.
% 
% Copyright 2017 United States Government as represented by the
% Administrator of the National Aeronautics and Space Administration.
% All Rights Reserved.
% 
% This file is available under the terms of the NASA Open Source Agreement
% (NOSA). You should have received a copy of this agreement with the
% Kepler source code; see the file NASA-OPEN-SOURCE-AGREEMENT.doc.
% 
% No Warranty: THE SUBJECT SOFTWARE IS PROVIDED "AS IS" WITHOUT ANY
% WARRANTY OF ANY KIND, EITHER EXPRESSED, IMPLIED, OR STATUTORY,
% INCLUDING, BUT NOT LIMITED TO, ANY WARRANTY THAT THE SUBJECT SOFTWARE
% WILL CONFORM TO SPECIFICATIONS, ANY IMPLIED WARRANTIES OF
% MERCHANTABILITY, FITNESS FOR A PARTICULAR PURPOSE, OR FREEDOM FROM
% INFRINGEMENT, ANY WARRANTY THAT THE SUBJECT SOFTWARE WILL BE ERROR
% FREE, OR ANY WARRANTY THAT DOCUMENTATION, IF PROVIDED, WILL CONFORM
% TO THE SUBJECT SOFTWARE. THIS AGREEMENT DOES NOT, IN ANY MANNER,
% CONSTITUTE AN ENDORSEMENT BY GOVERNMENT AGENCY OR ANY PRIOR RECIPIENT
% OF ANY RESULTS, RESULTING DESIGNS, HARDWARE, SOFTWARE PRODUCTS OR ANY
% OTHER APPLICATIONS RESULTING FROM USE OF THE SUBJECT SOFTWARE.
% FURTHER, GOVERNMENT AGENCY DISCLAIMS ALL WARRANTIES AND LIABILITIES
% REGARDING THIRD-PARTY SOFTWARE, IF PRESENT IN THE ORIGINAL SOFTWARE,
% AND DISTRIBUTES IT "AS IS."
%
% Waiver and Indemnity: RECIPIENT AGREES TO WAIVE ANY AND ALL CLAIMS
% AGAINST THE UNITED STATES GOVERNMENT, ITS CONTRACTORS AND
% SUBCONTRACTORS, AS WELL AS ANY PRIOR RECIPIENT. IF RECIPIENT'S USE OF
% THE SUBJECT SOFTWARE RESULTS IN ANY LIABILITIES, DEMANDS, DAMAGES,
% EXPENSES OR LOSSES ARISING FROM SUCH USE, INCLUDING ANY DAMAGES FROM
% PRODUCTS BASED ON, OR RESULTING FROM, RECIPIENT'S USE OF THE SUBJECT
% SOFTWARE, RECIPIENT SHALL INDEMNIFY AND HOLD HARMLESS THE UNITED
% STATES GOVERNMENT, ITS CONTRACTORS AND SUBCONTRACTORS, AS WELL AS ANY
% PRIOR RECIPIENT, TO THE EXTENT PERMITTED BY LAW. RECIPIENT'S SOLE
% REMEDY FOR ANY SUCH MATTER SHALL BE THE IMMEDIATE, UNILATERAL
% TERMINATION OF THIS AGREEMENT.
%

\section{Summary Metrics}

\subsection{Black Level Metric}

\ifdefined \meanBlackLevelVariation

\begin{center}
  \includegraphics[width=\imagewidth]{\meanBlackLevelVariation.png}
\end{center}

\meanBlackLevelVariationCaption

Open \url{\meanBlackLevelVariation.fig}

\else
No figure named
Mean\_black\_level\_variation\_over\_the\_focal\_plane\_in\_ADU.fig is
available.
\fi
\clearpage

\ifdefined \blackLevelMetricVariationByModout

\begin{center}
  \includegraphics[width=\imagewidth]{\blackLevelMetricVariationByModout.png}
\end{center}

\blackLevelMetricVariationByModoutCaption

Open \url{\blackLevelMetricVariationByModout.fig}

\else
No figure named
Black\_level\_metric\_variation\_across\_the\_focal\_plane\_over\_*\_modouts.fig is
available.
\fi
\clearpage

\ifdefined \blackLevelMetricVariationByCadence

\begin{center}
  \includegraphics[width=\imagewidth]{\blackLevelMetricVariationByCadence.png}
\end{center}

The black level metric is computed at the end of the calibration
process which consists of the following steps:

\begin{enumerate}
\item
  Subtract requantization table offset from the raw black collateral
  pixels.
\item
  Add mean black level value per read multiplied by the number of
  exposures per long cadence.
\item
  Subtract black 2D.
\item
  Bin the collateral black pixels from several columns into one column.
\item
  Fit a polynomial over the calibrated, binned black pixels from step 4.
\item
  Evaluate the polynomial over the available rows to obtain the
  calibrated black pixels.
\end{enumerate}

The black metric is calculated as the mean value of calibrated black
pixels from step 6. Its units are in ADU.

In this plot, black metric variation over the focal plane over several
cadences is plotted along with the uncertainties.

In general, if there is more than one cadence, one would expect to see
plots for all the cadences merge into one plot. If many plots are
seen, then it indicates strong variations across cadences for the same
module output. This would warrant an in-depth analysis of the
intermediate products left in the Matlab workspace.

Open \url{\blackLevelMetricVariationByCadence.fig}

\else
No figure named
Black\_level\_metric\_variation\_across\_the\_focal\_plane\_over\_*\_cadences.fig is
available.
\fi
\clearpage

\subsection{Smear Level Metric}

\ifdefined \meanSmearLevelVariation

\begin{center}
  \includegraphics[width=\imagewidth]{\meanSmearLevelVariation.png}
\end{center}

\meanSmearLevelVariationCaption

Open \url{\meanSmearLevelVariation.fig}

\else
No figure named
Mean\_smear\_level\_variation\_over\_the\_focal\_plane\_in\_photoelectrons.fig is
available.
\fi
\clearpage

\ifdefined \smearLevelMetricVariationByModout

\begin{center}
  \includegraphics[width=\imagewidth]{\smearLevelMetricVariationByModout.png}
\end{center}

\smearLevelMetricVariationByModoutCaption

Open \url{\smearLevelMetricVariationByModout.fig}

\else
No figure named
Smear\_level\_metric\_variation\_across\_the\_focal\_plane\_over\_*\_modouts.fig is
available.
\fi
\clearpage

\ifdefined \smearLevelMetricVariationByCadence

\begin{center}
  \includegraphics[width=\imagewidth]{\smearLevelMetricVariationByCadence.png}
\end{center}

The smear level metric is computed at the end of the calibration
process which consists of the following steps:

\begin{enumerate}
\item
  Subtract requantization table offset from the raw masked/virtual smear
  collateral pixels.
\item
  Add mean black level value per read multiplied by the number of
  exposures per long cadence.
\item
  Subtract black 2D.
\item
  Bin the collateral masked/virtual smear pixels from several rows into
  one row each.
\item
  Correct for gain.
\item
  Correct for undershoot.
\item
  Estimate the smear from masked/virtual smear pixels.
\end{enumerate}

Smear metric is calculated as the median value of calibrated smear
pixels from step 7. Its units are in photoelectrons.

In this plot, smear metric variation over the focal plane over several
cadences is plotted along with the uncertainties.

In general, if there is more than one cadence, one would expect to see
plots for all the cadences merge into one plot. If many plots are
seen, then it indicates strong variations across cadences for the same
module output. This would warrant an in-depth analysis of the
intermediate products left in the Matlab workspace.

Open \url{\smearLevelMetricVariationByCadence.fig}

\else
No figure named
Smear\_level\_metric\_variation\_across\_the\_focal\_plane\_over\_*\_cadences.fig is
available.
\fi
\clearpage

\subsection{Dark Current Metric}

\ifdefined \meanDarkCurrentLevelVariation

\begin{center}
  \includegraphics[width=\imagewidth]{\meanDarkCurrentLevelVariation.png}
\end{center}

\meanDarkCurrentLevelVariationCaption

Open \url{\meanDarkCurrentLevelVariation.fig}

\else
No figure named
Mean\_dark\_current\_level\_variation\_over\_the\_focal\_plane\_in\_photoelectrons\_per\_sec\_per\_exposure.fig is
available.
\fi
\clearpage

\ifdefined \darkCurrentMetricVariationByModout

\begin{center}
  \includegraphics[width=\imagewidth]{\darkCurrentMetricVariationByModout.png}
\end{center}

\darkCurrentMetricVariationByModoutCaption

Open \url{\darkCurrentMetricVariationByModout.fig}

\else
No figure named
Dark\_current\_metric\_variation\_across\_the\_focal\_plane\_over\_*\_modouts.fig is
available.
\fi
\clearpage

\ifdefined \darkCurrentMetricVariationByCadence

\begin{center}
  \includegraphics[width=\imagewidth]{\darkCurrentMetricVariationByCadence.png}
\end{center}

The dark current metric is computed at the end of the calibration
process which consists of the following steps:

\begin{enumerate}
\item
  Subtract requantization table offset from the raw masked/virtual smear
  collateral pixels.
\item
  Add mean black level value per read multiplied by the number of
  exposures per long cadence.
\item
  Subtract black 2D.
\item
  Bin the collateral masked/virtual smear pixels from several rows into
  one row each.
\item
  Correct for gain.
\item
  Correct for undershoot.
\item
  Estimate the dark currents from smear from the calibrated
  masked/virtual smear pixels.
\end{enumerate}

Dark current metric is calculated as the median value of dark current
from step 7. Its units are in photoelectrons per second per exposure.

In this plot, dark current metric variation over the focal plane over
several over several cadences is plotted along with the uncertainties.

In general, if there is more than one cadence, one would expect to see
plots for all the cadences merge into one plot. If many plots are
seen, then it indicates strong variations across cadences for the same
module output. This would warrant an in-depth analysis of the
intermediate products left in the Matlab workspace.

Open \url{\darkCurrentMetricVariationByCadence.fig}

\else
No figure named
Dark\_current\_metric\_variation\_across\_the\_focal\_plane\_over\_*\_cadences.fig is
available.
\fi
\clearpage

\subsection{Background Level Metric}

\ifdefined \meanBackgroundLevelVariation

\begin{center}
  \includegraphics[width=\imagewidth]{\meanBackgroundLevelVariation.png}
\end{center}

\meanBackgroundLevelVariationCaption

Open \url{\meanBackgroundLevelVariation.fig}

\else
No figure named
Mean\_background\_level\_variation\_over\_the\_focal\_plane\_in\_photoelectrons.fig is
available.
\fi
\clearpage

\ifdefined \backgroundLevelMetricVariationByModout

\begin{center}
  \includegraphics[width=\imagewidth]{\backgroundLevelMetricVariationByModout.png}
\end{center}

\backgroundLevelMetricVariationByModoutCaption

Open \url{\backgroundLevelMetricVariationByModout.fig}

\else
No figure named
Background\_level\_metric\_variation\_across\_the\_focal\_plane\_over\_*\_modouts.fig is
available.
\fi
\clearpage

\ifdefined \backgroundLevelMetricVariationByCadence

\begin{center}
  \includegraphics[width=\imagewidth]{\backgroundLevelMetricVariationByCadence.png}
\end{center}

The background level metric is computed at the end of the calibration
process which consists of the following steps:

\begin{enumerate}
\item 
  Subtract requantization table offset from the raw background
  collateral pixels.
\item 
  Add mean black level value per read multiplied by the number of
  exposures per long cadence.
\item 
  Subtract black 2D.
\item 
  Correct for gain.
\item 
  Correct for undershoot.
\item 
  Collect additional background pixels from the aperture assigned to the
  targets.
\item 
  Correct for flat field.
\item 
  Remove outliers in the background flux.
\end{enumerate}

Background level metric is calculated as the median value of
background level from step 8. Its units are in photoelectrons.

In this plot, background level metric variation over the focal plane
over several cadences is plotted along with the uncertainties.

In general, if there is more than one cadence, one would expect to see
plots for all the cadences merge into one plot. If many plots are
seen, then it indicates strong variations across cadences for the same
module output. This would warrant an in-depth analysis of the
intermediate products left in the Matlab workspace.

Open \url{\backgroundLevelMetricVariationByCadence.fig}

\else
No figure named
Background\_level\_metric\_variation\_across\_the\_focal\_plane\_over\_*\_cadences.fig is
available.
\fi
\clearpage

\subsection{Dynamic Range Metric}

\ifdefined \meanDynamicRangeVariation

\begin{center}
  \includegraphics[width=\imagewidth]{\meanDynamicRangeVariation.png}
\end{center}

\meanDynamicRangeVariationCaption

Open \url{\meanDynamicRangeVariation.fig}

\else
No figure named
Mean\_dynamic\_range\_variation\_over\_the\_focal\_plane\_in\_ADU.fig is
available.
\fi
\clearpage

\ifdefined \dynamicRangeMetricVariationByModout

\begin{center}
  \includegraphics[width=\imagewidth]{\dynamicRangeMetricVariationByModout.png}
\end{center}

\dynamicRangeMetricVariationByModoutCaption

Open \url{\dynamicRangeMetricVariationByModout.fig}

\else
No figure named
Dynamic\_range\_metric\_variation\_across\_the\_focal\_plane\_over\_*\_modouts.fig is
available.
\fi
\clearpage

\ifdefined \dynamicRangeMetricVariationByCadence

\begin{center}
  \includegraphics[width=\imagewidth]{\dynamicRangeMetricVariationByCadence.png}
\end{center}

The dynamic range metric is computed as follows:

\begin{verbatim}
  (max(all pixels) - min(all pixels))/numberOfExposuresPerLongCadence
\end{verbatim}

Its units are in ADU per exposure.

In this plot, dynamic range metric variation over the focal plane over
several cadences is plotted along with the uncertainties.

In general, if there is more than one cadence, one would expect to see
plots for all the cadences merge into one plot. If many plots are
seen, then it indicates strong variations across cadences for the same
module output. This would warrant an in-depth analysis of the
intermediate products left in the Matlab workspace.

Open \url{\dynamicRangeMetricVariationByCadence.fig}

\else
No figure named
Dynamic\_range\_metric\_variation\_across\_the\_focal\_plane\_over\_*\_cadences.fig is
available.
\fi
\clearpage

\subsection{Brightness Metric}

\ifdefined \meanBrightnessMetricVariation

\begin{center}
  \includegraphics[width=\imagewidth]{\meanBrightnessMetricVariation.png}
\end{center}

\meanBrightnessMetricVariationCaption

Open \url{\meanBrightnessMetricVariation.fig}

\else
No figure named
Mean\_brightness\_metric\_variation\_over\_the\_focal\_plane\_in\_unitless\_ratio.fig is
available.
\fi
\clearpage

\ifdefined \brightnessMetricVariationByModout

\begin{center}
  \includegraphics[width=\imagewidth]{\brightnessMetricVariationByModout.png}
\end{center}

\brightnessMetricVariationByModoutCaption

Open \url{\brightnessMetricVariationByModout.fig}

\else
No figure named
Brightness\_metric\_variation\_across\_the\_focal\_plane\_over\_*\_modouts.fig is
available.
\fi
\clearpage

\ifdefined \brightnessMetricVariationByCadence

\begin{center}
  \includegraphics[width=\imagewidth]{\brightnessMetricVariationByCadence.png}
\end{center}

The brightness metric metric is computed as follows:

\begin{enumerate}
\item 
  Calculate flux of each star for each cadence using simple aperture
  photometry.
\item 
  Compute corrected flux by dividing the flux from step 1 by flux
  fraction in aperture (computed by TAD).
\item 
  Normalize the corrected flux by dividing the flux from step 2 by
  expected flux. Expected flux is computed as:

\begin{verbatim}
  expectedFlux = standardMag12Flux * ccdExposureTime 
  * numberOfExposuresPerLongCadence * mag2b(starMag-12)
\end{verbatim}

\item 
  Derive flux metric for each cadence as the robust mean brightness of
  all targets in the current module.
\end{enumerate}

In this plot, brightness metric metric variation over the focal plane
over several cadences is plotted along with the uncertainties.

In general, if there is more than one cadence, one would expect to see
plots for all the cadences merge into one plot. If many plots are
seen, then it indicates strong variations across cadences for the same
module output. This would warrant an in-depth analysis of the
intermediate products left in the Matlab workspace.

Open \url{\brightnessMetricVariationByCadence.fig}

\else
No figure named
Brightness\_metric\_variation\_across\_the\_focal\_plane\_over\_*\_cadences.fig is
available.
\fi
\clearpage

\subsection{Centroid Row Metric}

\ifdefined \meanCentroidRowMetricVariation

\begin{center}
  \includegraphics[width=\imagewidth]{\meanCentroidRowMetricVariation.png}
\end{center}

\meanCentroidRowMetricVariationCaption

Open \url{\meanCentroidRowMetricVariation.fig}

\else
No figure named
Mean\_centroid\_row\_metric\_variation\_over\_the\_focal\_plane\_in\_pixels.fig is
available.
\fi
\clearpage

\ifdefined \centroidRowMetricVariationByModout

\begin{center}
  \includegraphics[width=\imagewidth]{\centroidRowMetricVariationByModout.png}
\end{center}

\centroidRowMetricVariationByModoutCaption

Open \url{\centroidRowMetricVariationByModout.fig}

\else
No figure named
Centroid\_row\_metric\_variation\_across\_the\_focal\_plane\_over\_*\_modouts.fig is
available.
\fi
\clearpage

\ifdefined \centroidRowMetricVariationByCadence

\begin{center}
  \includegraphics[width=\imagewidth]{\centroidRowMetricVariationByCadence.png}
\end{center}

The centroid row metric is calculated as follows: 

\begin{enumerate}
\item 
  Use the recently computed pointing to compute the predicted centroid
  row, column positions of target stars (use ra\_dec\_2\_pix on ra, dec
  of stars).
\item 
  Compute centroid row metric as the robust mean of \{predicted centroid
  row positions - measured centroid row positions\}.
\end{enumerate}

In this plot, centroid row metric metric variation over the focal
plane over several cadences is plotted along with the uncertainties.

In general, if there is more than one cadence, one would expect to see
plots for all the cadences merge into one plot. If many plots are
seen, then it indicates strong variations across cadences for the same
module output. This would warrant an in-depth analysis of the
intermediate products left in the Matlab workspace.

Open \url{\centroidRowMetricVariationByCadence.fig}

\else
No figure named
Centroid\_row\_metric\_variation\_across\_the\_focal\_plane\_over\_*\_cadences.fig is
available.
\fi
\clearpage

\subsection{Centroid Column Metric}

\ifdefined \meanCentroidColumnMetricVariation

\begin{center}
  \includegraphics[width=\imagewidth]{\meanCentroidColumnMetricVariation.png}
\end{center}

\meanCentroidColumnMetricVariationCaption

Open \url{\meanCentroidColumnMetricVariation.fig}

\else
No figure named
Mean\_centroid\_column\_metric\_variation\_over\_the\_focal\_plane\_in\_pixels.fig is
available.
\fi
\clearpage

\ifdefined \centroidColumnMetricVariationByModout

\begin{center}
  \includegraphics[width=\imagewidth]{\centroidColumnMetricVariationByModout.png}
\end{center}

\centroidColumnMetricVariationByModoutCaption

Open \url{\centroidColumnMetricVariationByModout.fig}

\else
No figure named
Centroid\_column\_metric\_variation\_across\_the\_focal\_plane\_over\_*\_modouts.fig is
available.
\fi
\clearpage

\ifdefined \centroidColumnMetricVariationByCadence

\begin{center}
  \includegraphics[width=\imagewidth]{\centroidColumnMetricVariationByCadence.png}
\end{center}

The centroid row metric is calculated as follows: 

\begin{enumerate}
\item 
  Use the recently computed pointing to compute the predicted centroid
  row, column positions of target stars (use ra\_dec\_2\_pix on ra, dec
  of stars).
\item 
  Compute centroid row metric as the robust mean of \{predicted centroid
  row positions - measured centroid row positions\}.
\end{enumerate}

In this plot, centroid row metric metric variation over the focal
plane over several cadences is plotted along with the uncertainties.

In general, if there is more than one cadence, one would expect to see
plots for all the cadences merge into one plot. If many plots are
seen, then it indicates strong variations across cadences for the same
module output. This would warrant an in-depth analysis of the
intermediate products left in the Matlab workspace.

Open \url{\centroidColumnMetricVariationByCadence.fig}

\else
No figure named
Centroid\_column\_metric\_variation\_across\_the\_focal\_plane\_over\_*\_cadences.fig is
available.
\fi
\clearpage

\subsection{Encircled Energy Metric}

\ifdefined \meanEncircledEnergyVariation

\begin{center}
  \includegraphics[width=\imagewidth]{\meanEncircledEnergyVariation.png}
\end{center}

\meanEncircledEnergyVariationCaption

Open \url{\meanEncircledEnergyVariation.fig}

\else
No figure named
Mean\_encircled\_energy\_variation\_over\_the\_focal\_plane\_in\_pixels.fig is
available.
\fi
\clearpage

\ifdefined \encircledEnergyMetricVariationByModout

\begin{center}
  \includegraphics[width=\imagewidth]{\encircledEnergyMetricVariationByModout.png}
\end{center}

\encircledEnergyMetricVariationByModoutCaption

Open \url{\encircledEnergyMetricVariationByModout.fig}

\else
No figure named
Encircled\_energy\_metric\_variation\_across\_the\_focal\_plane\_over\_*\_modouts.fig is
available.
\fi
\clearpage

\ifdefined \encircledEnergyMetricVariationByCadence

\begin{center}
  \includegraphics[width=\imagewidth]{\encircledEnergyMetricVariationByCadence.png}
\end{center}

\encircledEnergyMetricVariationByCadenceCaption

Open \url{\encircledEnergyMetricVariationByCadence.fig}

\else
No figure named
Encircled\_energy\_metric\_variation\_across\_the\_focal\_plane\_over\_*\_cadences.fig is
available.
\fi
\clearpage

\subsection{Plate Scale Metric}

\ifdefined \meanPlateScaleMetricVariation

\begin{center}
  \includegraphics[width=\imagewidth]{\meanPlateScaleMetricVariation.png}
\end{center}

\meanPlateScaleMetricVariationCaption

Open \url{\meanPlateScaleMetricVariation.fig}

\else
No figure named
Mean\_plate\_scale\_metric\_variation\_over\_the\_focal\_plane\_in\_pixels.fig
is available.
\fi
\clearpage

\ifdefined \plateScaleMetricVariationByModout

\begin{center}
  \includegraphics[width=\imagewidth]{\plateScaleMetricVariationByModout.png}
\end{center}

\plateScaleMetricVariationByModoutCaption

Open \url{\plateScaleMetricVariationByModout.fig}

\else
No figure named
Plate\_scale\_metric\_variation\_across\_the\_focal\_plane\_over\_*\_modouts.fig
is available.
\fi
\clearpage

\ifdefined \plateScaleMetricVariationByCadence

\begin{center}
  \includegraphics[width=\imagewidth]{\plateScaleMetricVariationByCadence.png}
\end{center}

The plate scale metric is calculated as follows. For each cadence, 

\begin{enumerate}
\item 
  Measure centroid positions \{row, col\}.
\item 
  Use the computed pointing from attitude solution for all the reference
  pixel time stamps.
\item 
  Invoke pix\_2\_ra\_dec with the measured centroids, cadence time
  stamps, computed pointing, and the velocity aberration flag turned on
  to get their predicted \{ra, dec\}.
\item 
  Estimate w from the affine transformation (ratios of distances between
  points are preserved) defined by \texttt{A*W = Y} where:
  \begin{list}{o}{
      \setlength{\parsep}{0cm}
      \setlength{\partopsep}{0cm}
      \setlength{\itemsep}{0cm}}
  \item[{\bf A}]
    is the design matrix of size [nStars X 3], with the first two
    columns being ra, dec of stars from the catalog and the third column
    being a column of 1's, and
  \item[{\bf Y}]
    is a matrix of size [nStars X 2] with the two columns being predicted
    ra, dec of stars obtained by a transformation of measured centroids,
    and
  \item[{\bf W}]
    is a matrix of size [3 X 2] with the first column containing the
    coefficients a, b, and c and the second column containing the
    coefficients d,e, and f.
  \end{list}
\end{enumerate}

The plate scale is computed as the determinant of the matrix composed
of the the first two rows of W and is equal to (a*e - b*d).

In this plot, plate scale metric metric variation over the focal plane
over several cadences is plotted along with the uncertainties.

In general, if there is more than one cadence, one would expect to see
plots for all the cadences merge into one plot. If many plots are
seen, then it indicates strong variations across cadences for the same
module output. This would warrant an in-depth analysis of the
intermediate products left in the Matlab workspace.

Open \url{\plateScaleMetricVariationByCadence.fig}

\else
No figure named
Plate\_scale\_metric\_variation\_across\_the\_focal\_plane\_over\_*\_cadences.fig
is available.
\fi
\clearpage
